\documentclass[12pt,draft]{article}
\usepackage[utf8x]{inputenc}
\usepackage[english,russian]{babel}
\usepackage{cmap}
\usepackage{amsthm}
\usepackage{yfonts}

\begin{document}

\section*{Основные определения}

\newtheorem{definition}{Определение}
\begin{definition}
    Марковской цепью (Markov chain) называется пара 
    $(\mathbf{\cal S}, \mathbf{\cal P})$, где

    \begin{enumerate}
        \item $\mathbf{\cal S}$ -- множество состояний.
        \item $\mathbf{\cal P}$ -- вероятность переходов $\{p(s_{t+1}|s_{t}) 
        | t \in \{0, 1, 2, ...\}, s_{t}, s_{t+1} \in \mathbf{\cal S}\}$.
     \end{enumerate}


\end{definition}

Что-то про теорему
\begin{definition}
    
    Марковская цепь называется {\bf однородной}
    или {\bf стационарной}, если вероятность перехода не зависит от времени:

    $$\forall t: p(s_{t+1} | s_{t}) = p(s_{1} | s_{0})$$


\end{definition}

\end{document}

